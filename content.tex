%----------------------------------------------------------------------------------------
%	EDUCATION SECTION
%----------------------------------------------------------------------------------------
\begin{rSection}{Education}

{\bf University of Michigan} \hfill {\em Ann Arbor, MI} 
\\ PhD Physics and Scientific Computing Candidate, \textbf{GPA: 3.96} \hfill {\em Expected June 2024}
\vspace*{-0.1cm}
%

{\bf Stanford University} \hfill {\em Stanford, CA} 
\\ MS Computer Science, \textbf{GPA: 3.74} \hfill {\em September 2018 - June 2019}

\vspace*{-0.1cm}
%
\begin{itemize}[itemsep=-0.3em] %compress item separation for aesthetics. 

    \item 
    \textbf{Research in Statistics and Cosmology}: ``Effects of Overlapping Sources on Cosmic Shear Estimation:  Statistical Sensitivity and Pixel-Noise Bias''
\end{itemize}

%
BS Physics with Honors \& Minor in Statistics, \textbf{GPA: 3.86} \hfill {\em September 2014 - June 2018}
%
\vspace*{-0.1cm}
\begin{itemize}[itemsep=-0.25em] 

    \item 
    \textbf{Honors Thesis}: ``No escape: light waves in AdS" (\href{https://purl.stanford.edu/vf208qp2190}{Link})

\end{itemize}
%
\vspace*{-0.1cm}
\end{rSection}

%----------------------------------------------------------------------------------------
%	RESEARCH EXPERIENCE SECTION
%----------------------------------------------------------------------------------------

\begin{rSection}{Research Experience}

\begin{rSubsection}{Probabilistic Modeling with Machine Learning in Cosmology}{University of Michigan, MI}{Advisor: Jeffrey Regier (Statistics)}{October 2019 -- Present}
\item Lead maintainer and developer of \textit{\href{https://github.com/prob-ml/bliss}{BLISS}}, an open-source Python package designed to measure visually overlapping (blended) galaxies in state-of-the-art astronomical surveys.
\item Built probabilistic model to measure blended galaxy images using latest algorithms in variational inference and deep generative modeling.
\item Created pipeline to train, validate, and test machine learning algorithms on real astronomical images.
\end{rSubsection}

\begin{rSubsection}{Leadership in Open Source Software Development}{University of Michigan, MI}{Advisor: Camille Avestruz (Physics)}{June 2020 -- Present}
\item Lead maintainer and developer of the \textit{\href{https://github.com/LSSTDESC/BlendingToolKit}{BlendingToolKit}}, a software tool kit for evaluating performance metrics for detection, deblending and measurement algorithms, applied to images of blended galaxies.

\item Presented software tutorials at collaboration meetings, which recruited a team of contributors.

\item Led team to extend user interface, incorporate realistic galaxy simulations, and create additional tutorials and documentation.
\end{rSubsection}

\begin{rSubsection}{Computational Cosmology}{University of Michigan, MI}{Advisor: Camille Avestruz (Cosmology)}{September 2019 -- Present}
\item Using dark matter halo catalogs based on N-body simulations to tie together their dynamical history and snapshot properties. 
\item Created pipeline to easily extract dark matter halo present-day properties, merger tree information, and subhalo information for a random subset of haloes at fixed mass.
\item Designed and executed statistical model to predict present-day dark matter halo properties from its accretion histories.
\end{rSubsection}

\begin{rSubsection}{Observational Cosmology and Data Analysis}{Stanford, CA}{Advisor: Patricia Burchat (Cosmology)}{June 2015 -- April 2021}
\item Developed a statistical framework for weak gravitational lensing that provides a comprehensive analysis of shape measurement noise bias for blended galaxies.
\item Assessed the impact of blending on cosmic shear estimation for several astronomical surveys.
\item Publication accepted to the Journal of Cosmology and Astrophysics (JCAP). 
\end{rSubsection}

%------------------------------------------------
\begin{rSubsection}{Biostatistics}{Stanford, CA}{Advisor: Julia Palacios (Statistics and Biomedical Data Science)}{September 2018 -- June 2019}
\item Implemented efficient algorithms for calculating the likelihood of phylogenetic trees simulated from coalescent models.

\item Developed Bayesian statistical framework to calculate the probability of correct classification between two different population size histories for large sample sizes and loci. 

% difficulty of distinguishing between possible population size histories through phylogenetic trees. 

% \item Applied stochastic modeling techniques to predict population sizes from phylogenetic trees. 
% \item Tested different hypothesis about human expansion out of Africa using simulations and Bayesian statistics. 
% \item \red{Used Bayesian Statistics to link genealogies to poSomething about Bayesian statistics... }
\end{rSubsection}

%------------------------------------------------
\begin{rSubsection}{ Convex Optimization}{Lausanne, Switzerland}{Advisor: Nisheeth Vishnoi (Theoretical Computer Science)}{June 2018 -- September 2018}
\item Participated in Summer@EPFL CS program at the \'Ecole polytechnique f\'ed\'erale de Lausanne (EPFL).
\item Designed and executed a project at interface of optimization, cosmology, and Riemannian geometry.
\item Developed manifold optimization algorithms to measure galaxy shapes from surface brightness profiles.
\item Used non-convex optimization techniques to mathematically show the high efficiency of my algorithm.
\end{rSubsection}
%------------------------------------------------
\begin{rSubsection}{General Relativity and Field Theory Honors Thesis}{Stanford, CA}{Advisor: Eva Silverstein (Cosmology)}{June 2017 -- June 2018}
\item Developed a framework for understanding scattering processes in manifolds by combining insights from quantum scattering theory, differential geometry, and partial differential equations. 
\item Applied framework to successfully resolve paradox of light waves traveling in Anti-de Sitter space. 
\item Simulated complex wave scattering processes using Mathematica. 
\item Presented work as my undergraduate Honors Thesis to the Stanford Physics Undergraduate Committee and at the Stanford Symposium of Undergraduate Research (SURPS).                        
\end{rSubsection}
\end{rSection}


%----------------------------------------------------------------------------------------
%   TEACHING EXPERIENCE SECTION
%----------------------------------------------------------------------------------------
\begin{rSection}{TEACHING EXPERIENCE} 
%
\begin{rSubsection}{Statistics Teaching Assistant at the University of Michigan}{Ann Arbor, MI}{}{}
\item \textbf{Courses:}
%
    \begin{itemize}[itemsep=-0.4em]
        \vspace*{-0.5em}
        \item Statistics 507: Data Science and Analytics using Python \hfill \textit{August 2020 -- December 2020}
    \end{itemize}

\item Led discussions sections to help students understand Python's scientific computing stack, relational databases (SQL), and deep learning using \pytorch.
\item Designed and graded weekly programming assignments.
\item Planned and executed a \texttt{kaggle} competition as their final project.
%
\end{rSubsection} 
%
\begin{rSubsection}{Physics Teaching Assistant at the University of Michigan}{Ann Arbor, MI}{}{}
\item \textbf{Courses:}
%
    \begin{itemize}[itemsep=-0.4em]
        \vspace*{-0.5em}
        \item Physics 136: Physics for the Life Sciences Laboratory I \hfill \textit{September 2019 -- December 2019}
        \item Physics 141: Elementary Laboratory I \hfill \textit{January 2020 -- April 2020}
        \item Physics 453: Quantum Mechanics \hfill \textit{January 2022 -- April 2022}
    \end{itemize}
%
\item Guided students through a series of physics experiments including analysis of their measurements.
\item Facilitated group discussions and provided regular feedback on student's performance. 
\end{rSubsection} 
%
\begin{rSubsection}{Physics Teaching Assistant at Stanford University}{Stanford, CA}{}{}
\item \textbf{Courses:}
%
    \begin{itemize}[itemsep=-0.5em]
        \vspace*{-0.5em}
        \item Physics 21: Mechanics, Fluids, and Heat \hfill \textit{September 2018 -- December 2018}
        \item Physics 70: Foundations of Modern Physics \hfill \textit{September 2017 -- December 2017}
    \end{itemize}
%
\item Designed and graded weekly problem sets, quizzes, and exams. 
\item Lead weekly problem-solving sessions aimed at reinforcing student's understanding of lecture. 
%
\end{rSubsection} 
%
\textbf{EPASA}: Tutored middle school student in Math and English. \hfill {\em September 2016 -- June 2018}

\vspace*{-.15cm}
\textbf{Habla}: Tutored Stanford custodial staff in English 3 hours/week. \hfill {\em September 2014 -- June 2018}
\end{rSection}

\newpage

%----------------------------------------------------------------------------------------
%   HONORS AND AWARDS SECTION
%----------------------------------------------------------------------------------------
\begin{rSection}{Honors and Awards} \itemsep -3pt \vspace*{-.25cm}

\item \textbf{Science Communication Fellowship} -- University of Michigan Museum of Natural History \hfill{\em 2022}

\item \textbf{Computational and Data Science Fellowship} -- \\ ACM’s Special Interest Group on High Performance Computing (SIGHPC) \hfill {\em 2021}

\item \textbf{Graduate Fellowship} -- Michigan Institute for Computational Discovery \& Engineering  \hfill {\em 2021}

\item \textbf{Enabling Science Award} -- Large Synoptic Survey Telescope Corporation \hfill {\em 2016 \& 2021}

\item \textbf{Research Grant} -- Stanford Undergraduate Advising and Research \hfill {\em 2017}

\item \textbf{Bronze Medalist} -- 45th International Physics Olympiad \hfill {\em 2014}
\end{rSection}
%----------------------------------------------------------------------------------------
% PROGRAMMING SKILLS SECTION
%----------------------------------------------------------------------------------------
\begin{rSection}{SKILLS}
\vspace*{-0.3cm}
\item {\small $\bullet$} \textit{Python:} 7+ years of experience in using Python for coursework and several research projects, including comprehensive knowledge of its scientific computing stack: \texttt{numpy}, \texttt{scipy}, \texttt{scikit-learn}, \texttt{pandas}, \texttt{matplotlib}.

\item {\small $\bullet$} \textit{Machine Learning:} Extensive experience designing and testing neural networks in \pytorch, developing ML pipelines for complex science applications, and knowledge of cutting-edge algorithms including variational autoencoders and normalizing flows.

\item {\small $\bullet$} \textit{Other Programming Languages:} C/C++, \LaTeX, Mathematica, Unix shell, Git, R.

\item {\small $\bullet$} \textit{Languages}: Native Spanish fluency.
\end{rSection}


%----------------------------------------------------------------------------------------
% PUBLICATIONS
%----------------------------------------------------------------------------------------
\begin{rSection}{PUBLICATIONS}
%
\vspace*{-1em}
\item Sanchez, J., \textbf{Mendoza, I.}, Kirkby, D. P., Burchat, P. R., and LSST Dark Energy Science Collaboration, “Effects of overlapping sources on cosmic shear estimation: Statistical sensitivity and pixel-noise bias”, \textit{Journal of Cosmology and Astroparticle Physics}, vol. 2021, no. 7, 2021. doi:10.1088/1475-7516/2021/07/043.
\end{rSection}




%----------------------------------------------------------------------------------------
% PRESENTATIONS
%----------------------------------------------------------------------------------------
\begin{rSection}{PRESENTATIONS}
%
\vspace*{-1em}

\item \textit{Bayesian Light Source Separator (BLISS)}, Ismael Mendoza, Dark Energy Science Collaboration (DESC) Winter 2022 Virtual Meeting Poster Session, February 2022 (virtual)

\item \textit{BLISS Update}, Ismael Mendoza, DESC Blending Working Group. February 2022 (virtual)

\item \textit{Bayesian Light Source Separator (BLISS)}, Ismael Mendoza, Dark Energy Science Collaboration (DESC) Summer 2021 Virtual Meeting Poster Session, July 2021 (virtual)

\item  \textit{(Updated) Blending ToolKit Tutorial}, Ismael Mendoza, Thomas Sainrat, Dark Energy Science Collaboration (DESC) Summer 2021 Virtual Meeting, July 2021 (virtual)

\item \textit{Connecting the Properties of Dark Matter Haloes with Their Growth}, Ismael Mendoza, University of Michigan Clusters Group, Ann Arbor, MI. March 2021 (virtual)
%
\item \textit{Effects of overlapping sources on cosmic shear estimation: Statistical sensitivity and pixel-noise bias}, Javier Sanchez \& Ismael Mendoza, Collaboration-Wide Presentation for the Dark Energy Science Collaboration (DESC). February 2021 (virtual)
%
\item \textit{Bayesian Light Source Separator (BLISS)}, Ismael Mendoza, Dark Energy Science Collaboration (DESC) Summer 2020 Virtual Meeting Poster Session, Chicago, IL. July 2020 (virtual)
%
\item  \textit{Blending ToolKit Tutorial}, Ismael Mendoza, Dark Energy Science Collaboration (DESC) Summer 2020 Virtual Meeting, Chicago, IL. July 2020 (virtual)
%
\item \textit{The Blending Problem in Cosmology}, Ismael Mendoza, Physics Graduate Student Symposium 2020, Ann Arbor, MI. July 2020 (virtual)
%
\item \textit{BlendingToolKit: Walkthrough and Future Plans}, Ismael Mendoza, DESC Blending Working Group. July 2020 (virtual)
\end{rSection}

