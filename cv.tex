%%%%%%%%%%%%%%%%%%%%%%%%%%%%%%%%%%%%%%%%%
% Medium Length Professional CV
% LaTeX Template
% Version 2.0 (8/5/13)
%
% This template has been downloaded from:
% http://www.LaTeXTemplates.com
%
% Original author:
% Trey Hunner (http://www.treyhunner.com/)
%
% Important note:
% This template requires the resume.cls file to be in the same directory as the
% .tex file. The resume.cls file provides the resume style used for structuring the
% document.
%
%%%%%%%%%%%%%%%%%%%%%%%%%%%%%%%%%%%%%%%%%%%%%%%%

%----------------------------------------------------------------------------------------
%	PACKAGES AND OTHER DOCUMENT CONFIGURATIONS
%----------------------------------------------------------------------------------------

\documentclass{resume} % Use the custom resume.cls style

\usepackage[left=1in,top=1in,right=1in,bottom=1in]{geometry} % Document margins
\newcommand{\tab}[1]{\hspace{.2667\textwidth}\rlap{#1}}
\newcommand{\itab}[1]{\hspace{0em}\rlap{#1}}
\usepackage[symbol, hang, flushmargin]{footmisc}
\usepackage{hyperref}
\usepackage{footnotebackref}
\usepackage{fontawesome5} % icons
\hypersetup{
    colorlinks=true,
    urlcolor=blue,
}
%footnotes
\renewcommand{\thefootnote}{\fnsymbol{footnote}}
% other commands
\newcommand{\pytorch}[0]{\texttt{pytorch}}
\newcommand{\pl}[0]{\texttt{pytorch-lightning}}

%%%% CONTENT STARTS
\name{Ismael Mendoza}
%%% Website
\address{\faIcon{globe} \url{https://ismael-mendoza.github.io}}
\address{\faIcon{github} \url{https://github.com/ismael-mendoza}}
\address{\faIcon{envelope} ismael@umd.edu}
\begin{document}
%----------------------------------------------------------------------------------------
%	JOBS SECTION
%----------------------------------------------------------------------------------------
\begin{rSection}{Jobs}
{\bf Astronomy Department, University of Maryland} \hfill {\em College Park, MD} 
\\ Post-Doctoral Associate \hfill {\em August 2025 -- Present}
\end{rSection}

%----------------------------------------------------------------------------------------
%	EDUCATION SECTION
%----------------------------------------------------------------------------------------
\begin{rSection}{Education}

{\bf University of Michigan} \hfill {\em Ann Arbor, MI} 
\\ PhD Physics, \textbf{GPA: 3.96} \hfill {\em September 2019 -- August 2025}
\vspace*{-0.1cm}
%

{\bf Stanford University} \hfill {\em Stanford, CA} 
\\ MS Computer Science, \textbf{GPA: 3.74} \hfill {\em September 2018 - June 2019}

\vspace*{-0.1cm}
%
\begin{itemize}[itemsep=-0.3em] %compress item separation for aesthetics. 

    \item 
    \textbf{Research in Statistics and Cosmology}: ``Effects of Overlapping Sources on Cosmic Shear Estimation:  Statistical Sensitivity and Pixel-Noise Bias''
\end{itemize}

%
BS Physics with Honors \& Minor in Statistics, \textbf{GPA: 3.86} \hfill {\em September 2014 - June 2018}
%
\vspace*{-0.1cm}
\begin{itemize}[itemsep=-0.25em] 

    \item 
    \textbf{Honors Thesis}: ``No escape: light waves in AdS" (\href{https://purl.stanford.edu/vf208qp2190}{Link})

\end{itemize}
%
\vspace*{-0.1cm}
\end{rSection}

%----------------------------------------------------------------------------------------
%	RESEARCH EXPERIENCE SECTION
%----------------------------------------------------------------------------------------

\begin{rSection}{Research Experience}

\begin{rSubsection}{Bayesian Shear Inference via Forward Modeling}{Argonne National Laboratory, IL}{Matthew Becker (Weak Lensing)}{August 2024 -- June 2025}
\item Developed \href{https://github.com/GalSim-developers/JAX-GalSim}{JAX-GalSim}, a fully differentiable replacement to the popular galaxy image simulation package, GalSim. 
\item We leverage JAX-Galsim to develop a new Bayesian cosmic shear measurement algorithm, which uses GPUs and gradient-based MCMCs for inference.
\item Executed project as part of my \href{https://science.osti.gov/wdts/scgsr}{SCGSR Fellowship} at Argonne National Laboratory.
\end{rSubsection}

\begin{rSubsection}{Galaxy-Halo Connection in N-body Cosmological Simulations}{University of Michigan, MI}{Advisor: Camille Avestruz (Cosmology)}{September 2019 -- Present}
\item Used dark matter halo catalogs from N-body simulations to connect the dynamical history and present-day properties of haloes.
\item Created pipeline to extract dark matter halo present-day properties, merger tree information, and subhalo information for a random subset of haloes at fixed mass.
\item Designed and implemented statistical model to predict present-day dark matter halo properties from its accretion histories.
\item Developing extension to predict clustering of galaxies in hydrodynamical simulation based on dark matter-only properties. 
\end{rSubsection}

\begin{rSubsection}{Probabilistic Modeling with ML in Cosmology Surveys}{University of Michigan, MI}{Advisor: Jeffrey Regier (Statistics)}{October 2019 -- June 2024}
\item Maintained and developed \textit{\href{https://github.com/prob-ml/bliss}{BLISS}}, an open-source Python package designed to measure visually overlapping (blended) galaxies in state-of-the-art astronomical surveys.
\item Built probabilistic model to measure blended galaxy images using techniques from variational inference and deep generative modeling.
\item Created pipeline to train, validate, and test machine learning algorithms on real astronomical images.
\end{rSubsection}

\begin{rSubsection}{Leadership in Open Source Software Development}{University of Michigan, MI}{Advisor: Camille Avestruz (Physics)}{June 2020 -- June 2024}
\item Lead maintainer and developer of the \textit{\href{https://github.com/LSSTDESC/BlendingToolKit}{BlendingToolKit}}, a software tool kit for evaluating performance metrics for detection, deblending and measurement algorithms, applied to images of blended galaxies.
\item Presented software tutorials at collaboration meetings, which recruited a team of contributors.
\item Led team to extend user interface, incorporate realistic galaxy simulations, and create additional tutorials and documentation.
\end{rSubsection}

\begin{rSubsection}{Impact of Blending on Weak Lensing Measurements with Fisher Formalism}{Stanford, CA}{Advisor: Patricia Burchat (Cosmology)}{June 2015 -- April 2021}
\item Developed software package to measure the impact of galaxy-galaxy blending on shape measurement noise bias.
\item Applied the Fisher formalism to assess the impact of blending on cosmic shear estimation for several astronomical surveys.
\item Publication accepted to the Journal of Cosmology and Astrophysics (JCAP). 
\end{rSubsection}

%------------------------------------------------
\begin{rSubsection}{Biostatistics}{Stanford, CA}{Advisor: Julia Palacios (Statistics and Biomedical Data Science)}{September 2018 -- June 2019}
\item Implemented efficient algorithms for calculating the likelihood of phylogenetic trees simulated from coalescent models.

\item Developed Bayesian statistical framework to calculate the probability of correct classification between two different population size histories for large sample sizes and loci. 

% difficulty of distinguishing between possible population size histories through phylogenetic trees. 

% \item Applied stochastic modeling techniques to predict population sizes from phylogenetic trees. 
% \item Tested different hypothesis about human expansion out of Africa using simulations and Bayesian statistics. 
% \item \red{Used Bayesian Statistics to link genealogies to poSomething about Bayesian statistics... }
\end{rSubsection}

%------------------------------------------------
\begin{rSubsection}{Convex Optimization}{Lausanne, Switzerland}{Advisor: Nisheeth Vishnoi (Theoretical Computer Science)}{June 2018 -- September 2018}
\item Participated in Summer@EPFL CS program at the \'Ecole polytechnique f\'ed\'erale de Lausanne (EPFL).
\item Designed and executed a project at interface of optimization, cosmology, and Riemannian geometry.
\item Developed manifold optimization algorithms to measure galaxy shapes from surface brightness profiles.
\item Used non-convex optimization techniques to mathematically show the high efficiency of my algorithm.
\end{rSubsection}
%------------------------------------------------
\begin{rSubsection}{General Relativity and Field Theory Honors Thesis}{Stanford, CA}{Advisor: Eva Silverstein (Cosmology)}{June 2017 -- June 2018}
\item Developed a framework for understanding scattering processes in manifolds by combining insights from quantum scattering theory, differential geometry, and partial differential equations. 
\item Applied framework to successfully resolve paradox of light waves traveling in Anti-de Sitter space. 
\item Simulated complex wave scattering processes using Mathematica. 
\item Presented work as my undergraduate Honors Thesis to the Stanford Physics Undergraduate Committee and at the Stanford Symposium of Undergraduate Research (SURPS).                        
\end{rSubsection}
\end{rSection}


%----------------------------------------------------------------------------------------
%   TEACHING EXPERIENCE SECTION
%----------------------------------------------------------------------------------------
\begin{rSection}{TEACHING EXPERIENCE} 
%
\begin{rSubsection}{Course development at the University of Michigan}{Ann Arbor, MI}{}{}
\item \textbf{Courses:}
%
    \begin{itemize}[itemsep=-0.4em]
        \vspace*{-0.5em}
        \item Physics 104: Introduction to Python Programming \hfill \textit{July 2022}
    \end{itemize}

\item Developed course materials for this new course, as well as a midterm/final project.

\end{rSubsection} 
%
\begin{rSubsection}{Instructor, Summer Program in Quantitative Methods for Social Research}{Ann Arbor, MI}{}{}
\item \textbf{Courses:}
%
    \begin{itemize}[itemsep=-0.4em]
        \vspace*{-0.5em}
        \item Introduction to Python \hfill \textit{July 2022}
    \end{itemize}

\item Designed and executed a 10-day bootcamp to introduce Python to newcomers. 

\item Emphasized a hands-on approach using Jupyter notebooks and encouraged interaction during lectures.

\end{rSubsection} 
%
\begin{rSubsection}{Statistics Teaching Assistant at the University of Michigan}{Ann Arbor, MI}{}{}
\item \textbf{Courses:}
%
    \begin{itemize}[itemsep=-0.4em]
        \vspace*{-0.5em}
        \item Statistics 507: Data Science and Analytics using Python \hfill \textit{August 2020 -- December 2020}
    \end{itemize}

\item Led discussions sections to help students understand Python's scientific computing stack, relational databases (SQL), and deep learning using \pytorch.
\item Designed and graded weekly programming assignments.
\item Planned and executed a \texttt{kaggle} competition as their final project.
%
\end{rSubsection} 
%
\begin{rSubsection}{Physics Teaching Assistant at the University of Michigan}{Ann Arbor, MI}{}{}
\item \textbf{Courses:}
%
    \begin{itemize}[itemsep=-0.4em]
        \vspace*{-0.5em}
        \item Physics 136: Physics for the Life Sciences Laboratory I \hfill \textit{September 2019 -- December 2019}
        \item Physics 141: Elementary Laboratory I \hfill \textit{January 2020 -- April 2020}
        \item Physics 453: Quantum Mechanics \hfill \textit{January 2022 -- April 2022}
        \item Physics 505: Classical fields and Electromagnetism I \hfill \textit{September 2023 -- December 2023}
    \end{itemize}
%
\item Guided students through a series of physics experiments including analysis of their measurements.
\item Facilitated group discussions and provided regular feedback on student's performance. 
\end{rSubsection} 
%
\begin{rSubsection}{Physics Teaching Assistant at Stanford University}{Stanford, CA}{}{}
\item \textbf{Courses:}
%
    \begin{itemize}[itemsep=-0.5em]
        \vspace*{-0.5em}
        \item Physics 21: Mechanics, Fluids, and Heat \hfill \textit{September 2018 -- December 2018}
        \item Physics 70: Foundations of Modern Physics \hfill \textit{September 2017 -- December 2017}
    \end{itemize}
%
\item Designed and graded weekly problem sets, quizzes, and exams. 
\item Lead weekly problem-solving sessions aimed at reinforcing student's understanding of lecture. 
%
\end{rSubsection} 
%
\textbf{EPASA}: Tutored middle school student in Math and English. \hfill {\em September 2016 -- June 2018}

\vspace*{-.15cm}
\textbf{Habla}: Tutored Stanford custodial staff in English 3 hours/week. \hfill {\em September 2014 -- June 2018}
\end{rSection}

%----------------------------------------------------------------------------------------
% ROLES SECTION
%----------------------------------------------------------------------------------------
\begin{rSection}{Leadership Roles} \itemsep -3pt \vspace*{-.25cm}

\item{\textbf{Topical Team Lead for LSST DESC} (2020 - 2025): Manage scientific teams to develop software that accomplishes goals within the collaboration.}

\item{\textbf{Sprint Coordinator for LSST DESC} (2022-2023): Organize hackathons and tutorials for the Dark Energy Science Collaboration.}

\item{\textbf{Physics Graduate Council at the University of Michigan} (2022-2024): Represent the graduate student body at a department level, and organize social events to build community.}

\item{\textbf{Life in Graduate School Seminar Council at the University of Michigan} (2023): Organize bi-weekly seminars for Physics graduate students to learn about resources at the university that can help them during their PhD.}
\end{rSection}


%----------------------------------------------------------------------------------------
% SKILLS SECTION
%----------------------------------------------------------------------------------------
\begin{rSection}{SKILLS}
\vspace*{-0.3cm}
\item {\small $\bullet$} \textit{Python:} 10+ years of experience in using Python for coursework and several research projects, including comprehensive knowledge of its scientific computing stack: \texttt{numpy}, \texttt{scipy}, \texttt{scikit-learn},
\texttt{matplotlib}.

\item {\small $\bullet$} \textit{Machine Learning:} Extensive experience designing and testing neural networks in \pytorch, developing ML pipelines for complex science applications, and knowledge of cutting-edge algorithms such as variational autoencoders and normalizing flows.

\item {\small $\bullet$} \textit{Other Programming Languages:} C/C++, \LaTeX, Mathematica, Unix shell, Git   

\item {\small $\bullet$} \textit{Languages}: Native Spanish speaker
\end{rSection}

%----------------------------------------------------------------------------------------
%   HONORS AND AWARDS SECTION
%----------------------------------------------------------------------------------------
\begin{rSection}{Honors and Awards} \itemsep -3pt \vspace*{-.25cm}

\item \textbf{Science Graduate Student Research (SCGSR) award} -- U.S. DOE Office \hfill{\em 2024}

\item \textbf{Department Fellowship, Knoller Fund} -- UofM Physics Department \hfill{\em 2024}

\item \textbf{Leinweber Center for Theoretical Physics Summer Fellowship} -- UofM Physics Department \hfill{\em 2023}

\item \textbf{Walter F. Lewis Candidacy Fellowship} -- University of Michigan Physics Department \hfill{\em 2022}

\item \textbf{Science Communication Fellowship} -- University of Michigan Museum of Natural History \hfill{\em 2022}

\item \textbf{Computational and Data Science Fellowship} -- \\ ACM’s Special Interest Group on High Performance Computing (SIGHPC) \hfill {\em 2021}

\item \textbf{Graduate Fellowship} -- Michigan Institute for Computational Discovery \& Engineering  \hfill {\em 2021}

\item \textbf{Enabling Science Award} -- Large Synoptic Survey Telescope Corporation \hfill {\em 2016 \& 2021}

\item \textbf{Research Grant} -- Stanford Undergraduate Advising and Research \hfill {\em 2017}

\item \textbf{Bronze Medalist} -- 45th International Physics Olympiad \hfill {\em 2014}
\end{rSection}



%----------------------------------------------------------------------------------------
% PUBLICATIONS
%----------------------------------------------------------------------------------------
\begin{rSection}{PUBLICATIONS}
%
\vspace*{-1em}

\item \textbf{Mendoza, I.},\footref{ft:equal} Torchylo, A.,\footref{ft:equal} Sainrat, T., Guinot, A., Boucaud, A., Paillasa, M., Avestruz, C., Adari, P., Aubourg, E., Biswas, B., Buchanan, J., et al. (2024). ``The Blending ToolKit: A simulation framework for evaluation of galaxy detection and deblending''. arXiv preprint arXiv:2409.06986. Submitted to the Open Journal of Astrophysics.

\item \textbf{Mendoza, I.}, Mansfield, P., Wang, K., and Avestruz, C. (2023). ``MultiCAM: a multivariable framework for connecting the mass accretion history of haloes with their properties''. \textit{Monthly Notices of the Royal Astronomical Society}, 523(4), 6386-6400.

\item \textbf{Mendoza, I.}, Liu, R., Hansen, D., Zhao, Z., Pang, Z., Avestruz, C., Regier, J., for the LSST Dark Energy Collaboration, ``Simulation-Based Inference for Probabilistic Light Source Detection, Deblending, and Measurement''. Submitted to the Dark Energy Science Collaboration (DESC) for internal review.

\item Wang, M.,\footref{ft:equal} \textbf{Mendoza, I.},\footref{ft:equal}, Wang, C., Avestruz, C., and Regier, J. (2022), ``Statistical Inference for Coadded Astronomical Images''. arXiv preprint arXiv:2211.09300. Accepted to the \textit{Machine Learning and the Physical Sciences Workshop at the 36th conference on Neural Information Processing Systems (NeurIPS)}.

\item Hansen, D.,\footnote[1]{\label{ft:equal} Equal contribution} \textbf{Mendoza, I.},\footref{ft:equal} Liu, R., Pang, Z., Zhao, Z., Avestruz, C., and Regier, J. (2022). ``Scalable Bayesian Inference for Detection and Deblending in Astronomical Images''. arXiv preprint arXiv:2207.05642. Accepted to the \textit{ICML 2022 Workshop on Machine Learning for Astrophysics}.

\item Sanchez, J., \textbf{Mendoza, I.}, Kirkby, D. P., Burchat, P. R., for the LSST Dark Energy Science Collaboration (2021). “Effects of overlapping sources on cosmic shear estimation: Statistical sensitivity and pixel-noise bias”. \textit{Journal of Cosmology and Astroparticle Physics}, 2021(07), 043.

\end{rSection}

%----------------------------------------------------------------------------------------
% PRESENTATIONS
%----------------------------------------------------------------------------------------
\begin{rSection}{PRESENTATIONS}
%
\vspace*{-1em}

\item \textit{Mitigating the Blending Problem in Cosmology}, Ismael Mendoza, Invited talk at Astrocoffee, Department of Physics and Astronomy at the University of Pittsburgh, Pittsburgh, PA, October 2023

\item \textit{The Blending Problem in Cosmology}, Ismael Mendoza, Invited talk at the KIPAC Tea, Kavli Institute for Particle Astrophysics and Cosmology at Stanford University, Stanford, CA, July 2023

\item \textit{Bayesian Light Source Separator (BLISS): Probabilistic detection, deblending and measurement of astronomical light sources}, Ismael Mendoza, Invited talk at Statistical Challenges in Modern Astronomy VIII, Pennsylvania State University, State College, PA, June 2023

\item \textit{MultiCAM: A multivariable framework for connecting the mass accretion history of haloes with their properties}, Ismael Mendoza, Invited talk at the Baryon Pasting Collaboration Meeting, Yale University, New Haven, CT, May 2023

\item \textit{MultiCAM: A multivariable framework for connecting the mass accretion history of haloes with their properties}, Ismael Mendoza, The Co-evolution of the Cosmic Web and Galaxies across Cosmic Time Conference Poster Session, Kavli Institute for Theoretical Physics (KITP), Santa Barbara, CA, February 2023

\item \textit{Statistical Inference for Coadded Astronomical Images}, Mallory Wang and Ismael Mendoza, Machine Learning and the Physical Sciences Workshop at the 36th conference on Neural Information Processing Systems (NeurIPS 2022) Poster Session, New Orleans, LA, December 2022

\item \textit{Bayesian Light Source Separator (BLISS)}, Ismael Mendoza, Dark Energy Science Collaboration (DESC) Summer Meeting Poster Session, Chicago, IL., August 2022

\item \textit{Scalable Bayesian Inference for Detection and Deblending in Astronomical Images}, Ismael Mendoza, ICML 2022 Workshop on Machine Learning for Astrophysics Poster Session, Baltimore, MA, July 2022

\item \textit{Machine Learning in Cosmology}, Ismael Mendoza, Physics Graduate Student Symposium 2022, Ann Arbor, MI, June 2022

\item \textit{Updates on the Bayesian Light Source Separator (BLISS)}, Ismael Mendoza, Dark Energy Science Collaboration (DESC) Bayesian Pipelines Topical Team Telecon, April 2022 (virtual)

% \item \textit{Bayesian Light Source Separator (BLISS)}, Ismael Mendoza, Dark Energy Science Collaboration (DESC) Winter 2022 Virtual Meeting Poster Session, February 2022 (virtual)

% \item \textit{BLISS Update}, Ismael Mendoza, DESC Blending Working Group. February 2022 (virtual)

% \item \textit{Bayesian Light Source Separator (BLISS)}, Ismael Mendoza, Dark Energy Science Collaboration (DESC) Summer 2021 Virtual Meeting Poster Session, July 2021 (virtual)

% \item  \textit{(Updated) Blending ToolKit Tutorial}, Ismael Mendoza, Thomas Sainrat, Dark Energy Science Collaboration (DESC) Summer 2021 Virtual Meeting, July 2021 (virtual)

\item \textit{Connecting the Properties of Dark Matter Haloes with Their Growth}, Ismael Mendoza, University of Michigan Clusters Group, Ann Arbor, MI, March 2021 (virtual)
%
\item \textit{Effects of overlapping sources on cosmic shear estimation: Statistical sensitivity and pixel-noise bias}, Javier Sanchez \& Ismael Mendoza, Collaboration-Wide Presentation for the Dark Energy Science Collaboration (DESC). February 2021 (virtual)
%
% \item \textit{Bayesian Light Source Separator (BLISS)}, Ismael Mendoza, Dark Energy Science Collaboration (DESC) Summer 2020 Virtual Meeting Poster Session, Chicago, IL. July 2020 (virtual)
%
\item \textit{BlendingToolKit Tutorial}, Ismael Mendoza, Dark Energy Science Collaboration (DESC) Summer 2020 Virtual Meeting, Chicago, IL, July 2020 (virtual)
%
\item \textit{The Blending Problem in Cosmology}, Ismael Mendoza, Physics Graduate Student Symposium 2020, Ann Arbor, MI, July 2020 (virtual)
%
\item \textit{BlendingToolKit: Walkthrough and Future Plans}, Ismael Mendoza, DESC Blending Working Group. July 2020 (virtual)
\end{rSection}

%----------------------------------------------------------------------------------------
% Blog posts
%----------------------------------------------------------------------------------------

\begin{rSection}{BLOG POSTS}
%
\vspace*{-1em}

\item \textit{MathStatBites at SCMA8: Astro Image Processing is BLISS?}, Andrew Saydjari for MathStatBites, \url{https://mathstatbites.org/mathstatbites-at-scma8-astro-image-processing-is-bliss/}

\item \textit{The Crowded Cosmos: Effects of Blended Galaxies on Cosmic Shear}, Katya Gozman for AstroBites, \url{https://astrobites.org/2021/03/20/blended-galaxies-cosmic-shear/}

\end{rSection}


\end{document}
