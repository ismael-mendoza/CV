%%%%%%%%%%%%%%%%%%%%%%%%%%%%%%%%%%%%%%%%%
% Medium Length Professional CV
% LaTeX Template
% Version 2.0 (8/5/13)
%
% This template has been downloaded from:
% http://www.LaTeXTemplates.com
%
% Original author:
% Trey Hunner (http://www.treyhunner.com/)
%
% Important note:
% This template requires the resume.cls file to be in the same directory as the
% .tex file. The resume.cls file provides the resume style used for structuring the
% document.
%
%%%%%%%%%%%%%%%%%%%%%%%%%%%%%%%%%%%%%%%%%

%----------------------------------------------------------------------------------------
%	PACKAGES AND OTHER DOCUMENT CONFIGURATIONS
%----------------------------------------------------------------------------------------

\documentclass{resume} % Use the custom resume.cls style

\usepackage[left=1in,top=0.7in,right=1in,bottom=1in]{geometry} % Document margins


\name{Ismael Mendoza}
\address{(+1)~734~834~2794  \\ imendoza@umich.edu \\ \url{https://ismael-mendoza.github.io/} }
\begin{document}
%----------------------------------------------------------------------------------------
%	EDUCATION SECTION
%----------------------------------------------------------------------------------------
\begin{rSection}{Education}

{\bf University of Michigan} \hfill {\em Ann Arbor, MI} 
\\ PhD Physics and Scientific Computing, \textbf{GPA: 3.96} \hfill {\em Expected June 2024}
\vspace*{-0.1cm}
%

{\bf Stanford University} \hfill {\em Stanford, CA} 
\\ MS Computer Science, \textbf{GPA: 3.74} \hfill {\em September 2018 - June 2019}

BS Physics with Honors | Minor in Statistics, \textbf{GPA: 3.86} \hfill {\em September 2014 - June 2018}
%
\vspace*{-0.1cm}
\begin{itemize}[itemsep=-0.25em] 

    \item 
    \textbf{Honors Thesis}: ``No escape: light waves in AdS" (Link: \url{https://purl.stanford.edu/vf208qp2190})  

\end{itemize}
%
\vspace*{-0.1cm}
\end{rSection}
%----------------------------------------------------------------------------------------
%	WORK EXPERIENCE SECTION
%----------------------------------------------------------------------------------------

\begin{rSection}{Research Experience}

\begin{rSubsection}{Probabilistic Modeling with Machine Learning in Cosmology}{University of Michigan, MI}{Advisor: Jeffrey Regier (Statistics)}{October 2019 -- Present}
\item Building a fully probabilistic light source separation algorithm for state-of-the-art astronomical surveys.
\end{rSubsection}

\begin{rSubsection}{Computational Cluster Cosmology}{University of Michigan, MI}{Advisor: Camille Avestruz (Physics)}{September 2019 -- Present}
\item Using N-body simulations and statistics to understand the dynamical evolution of dark matter halos. 
\end{rSubsection}

\begin{rSubsection}{Observational Cosmology and Data Analysis}{Stanford, CA}{Advisor: Patricia Burchat (Physics)}{June 2015 -- Present}
\item Developed a novel statistical framework for weak gravitational lensing using large scale galaxy simulations at the Stanford's High Performance Computing cluster.
\end{rSubsection}

%------------------------------------------------
\begin{rSubsection}{Convex Optimization}{Lausanne, Switzerland}{Advisor: Nisheeth Vishnoi (Theoretical Computer Science)}{June 2018 -- September 2018}
\item Participated in Summer@EPFL CS program at the \'Ecole polytechnique f\'ed\'erale de Lausanne (EPFL).
\item Designed and executed a project at interface of optimization, cosmology, and Riemannian geometry.
\end{rSubsection}
\end{rSection}

%----------------------------------------------------------------------------------------
%   TEACHING EXPERIENCE SECTION
%----------------------------------------------------------------------------------------
\begin{rSection}{TEACHING EXPERIENCE} 

\begin{rSubsection}{GSI at University of Michigan, Statistics Department}{Ann Arbor, MI}{}{}
    \item Statistics 507: Data Science and Analytics using Python \hfill \textit{September 2020 -- December 2020}
\end{rSubsection} 
%
\begin{rSubsection}{GSI at University of Michigan, Physics Department}{Ann Arbor, MI}{}{}
    \item Physics 136: Physics for the Life Sciences Laboratory I \hfill \textit{September 2019 -- December 2019}
    \item Physics 141: Elementary Laboratory I \hfill \textit{January 2020 -- April 2020}
\end{rSubsection} 
%
\begin{rSubsection}{Teaching Assistant at Stanford Physics Department}{Stanford, CA}{}{}
    \item Physics 21: Mechanics, Fluids, and Heat \hfill \textit{September 2018 -- December 2018}
    \item Physics 70: Foundations of Modern Physics \hfill \textit{September 2017 -- December 2017}
\end{rSubsection} 
\end{rSection}

%----------------------------------------------------------------------------------------
%   HONORS AND AWARDS SECTION
%----------------------------------------------------------------------------------------
\begin{rSection}{Honors and Awards} \itemsep -3pt \vspace*{-.25cm}

\item \textbf{Computational and Data Science Fellowship} -- \\ ACM’s Special Interest Group on High Performance Computing (SIGHPC) \hfill {\em 2021}

\item \textbf{Graduate Fellowship} -- Michigan Institute for Computational Discovery \& Engineering  \hfill {\em 2021}

\item \textbf{Enabling Science Award} -- Large Synoptic Survey Telescope Corporation \hfill {\em 2016 \& 2021}

\item \textbf{Research Grant} -- Stanford Undergraduate Advising and Research \hfill {\em 2017}

\item \textbf{Bronze Medalist}: 45th International Physics Olympiad \hfill {\em 2014}
\end{rSection}
%----------------------------------------------------------------------------------------
% PROGRAMMING SKILLS SECTION
%----------------------------------------------------------------------------------------
\begin{rSection}{PROGRAMMING SKILLS}
%
Python, C/C++, \LaTeX, Mathematica, Unix shell, Git, R
\end{rSection}


%----------------------------------------------------------------------------------------
% PUBLICATIONS
%----------------------------------------------------------------------------------------
\begin{rSection}{PUBLICATIONS}
%
\vspace*{-1em}
\item Sanchez, J., \textbf{Mendoza, I.}, Kirkby, D. P., \& Burchat, P. R. (2021). \textit{Effects of overlapping sources on cosmic shear estimation: Statistical sensitivity and pixel-noise bias.} arXiv preprint arXiv:2103.02078. Accepted to the Journal of Cosmology and Astrophysics (JCAP).
\end{rSection}


%%%%%%%%%%%%%%%%%%%%%%%%%%%%%%%%%%%%%%%%%%%%%%%%%%%%%%%%%%%%%%%%%%%%%%%%%%%%%%%%%%%%%%%%%%%%%%%%%%%%%%%%%%%%%%%%%%%%%%
\end{document}

% \begin{rSubsection}{Teaching Assistant for }{Stanford, CA}{Professor: Roger Romani}{September 2018 -– December 2018}
% \item Worked 20 hours/week to help teach this Physics course to Stanford undergraduates.   
% \item Helped students with topics such as quantum mechanics and special relativity during lecture and office hours. 
% \item 
% \end{rSubsection} 