%%%%%%%%%%%%%%%%%%%%%%%%%%%%%%%%%%%%%%%%%
% Medium Length Professional CV
% LaTeX Template
% Version 2.0 (8/5/13)
%
% This template has been downloaded from:
% http://www.LaTeXTemplates.com
%
% Original author:
% Trey Hunner (http://www.treyhunner.com/)
%
% Important note:
% This template requires the resume.cls file to be in the same directory as the
% .tex file. The resume.cls file provides the resume style used for structuring the
% document.
%
%%%%%%%%%%%%%%%%%%%%%%%%%%%%%%%%%%%%%%%%%

%----------------------------------------------------------------------------------------
%	PACKAGES AND OTHER DOCUMENT CONFIGURATIONS
%----------------------------------------------------------------------------------------

\documentclass{resume} % Use the custom resume.cls style

\usepackage[left=1in,top=1in,right=1in,bottom=1in]{geometry} % Document margins
\newcommand{\tab}[1]{\hspace{.2667\textwidth}\rlap{#1}}
\newcommand{\itab}[1]{\hspace{0em}\rlap{#1}}


\name{Ismael Mendoza} % Your name
\address{(+1)~650~285~7620  \\ imendoza@umich.edu \\ \url{https://ismael-mendoza.github.io/} } % Your address
\address{GitHub: \url{https://github.com/ismael-mendoza} } % Your secondary addess (optional)
\begin{document}
%----------------------------------------------------------------------------------------
%	EDUCATION SECTION
%----------------------------------------------------------------------------------------
\begin{rSection}{Education}

{\bf University of Michigan} \hfill {\em Ann Arbor, MI} 
\\ PhD Physics and Scientific Computing, \textbf{GPA: 4.00} \hfill {\em Expected June 2024}
\vspace*{-0.1cm}
%

{\bf Stanford University} \hfill {\em Stanford, CA} 
\\ MS Computer Science, \textbf{GPA: 3.74} \hfill {\em September 2018 - June 2019}

\vspace*{-0.1cm}
%
\begin{itemize}[itemsep=-0.3em] %compress item separation for aesthetics. 

    \item 
    \textbf{Research in Statistics and Cosmology}: ``Olber’s Paradox Revisited – Effects of Overlapping Sources on Cosmic Shear Estimation:  Statistical Sensitivity and Pixel-Noise Bias''
\end{itemize}

%
BS Physics with Honors | Minor in Statistics, \textbf{GPA: 3.86} \hfill {\em September 2014 - June 2018}
%
\vspace*{-0.1cm}
\begin{itemize}[itemsep=-0.25em] 

    \item 
    \textbf{Honors Thesis}: ``No escape: light waves in AdS" (Link: \url{https://purl.stanford.edu/vf208qp2190})  

\end{itemize}
%
\vspace*{-0.1cm}
\end{rSection}

%----------------------------------------------------------------------------------------
%	RESEARCH EXPERIENCE SECTION
%----------------------------------------------------------------------------------------

\begin{rSection}{Research Experience}

\begin{rSubsection}{Probabilistic Modeling with Machine Learning in Cosmology}{University of Michigan, MI}{Advisor: Jeffrey Regier (Statistics)}{October 2019 -- Present}
\item Developing a probabilistic framework to measure visually overlapping (blended) galaxies in state-of-the-art astronomical surveys like LSST. 
\item Building a deep generative model to characterize blended galaxies using variational autoencoders. 
\end{rSubsection}

\begin{rSubsection}{Computational Cosmology}{University of Michigan, MI}{Advisor: Camille Avestruz (Cosmology)}{September 2019 -- Present}
\item Using dark matter halo catalogs based on N-body simulations and a variety of statistical approaches to tie together dynamical history and snapshot properties of halos. 
\end{rSubsection}

\begin{rSubsection}{Observational Cosmology and Data Analysis}{Stanford, CA}{Advisor: Patricia Burchat (Cosmology)}{June 2015 -- Present}
\item Developed a statistical framework for weak gravitational lensing that provides a comprehensive analysis of shape measurement noise bias for blended galaxies.
\item Performed large scale galaxy simulations and measurements using Stanford's High Performance Computing cluster.
\item Presented my work in poster fairs and in several of the Dark Energy Science Collaboration's meetings. 
\item Currently preparing journal article for submission to the Journal of Cosmology and Astroparticle Physics that assesses the impact of blending on cosmic shear estimation for various astronomical surveys. 
\end{rSubsection}

%------------------------------------------------
\begin{rSubsection}{Biostatistics}{Stanford, CA}{Advisor: Julia Palacios (Statistics and Biomedical Data Science)}{September 2018 -- June 2019}
\item Implemented efficient algorithms for calculating the likelihood of phylogenetic trees simulated from coalescent models.

\item Developed Bayesian statistical framework to calculate the probability of correct classification between two different population size histories for large sample sizes and loci. 

% difficulty of distinguishing between possible population size histories through phylogenetic trees. 

% \item Applied stochastic modeling techniques to predict population sizes from phylogenetic trees. 
% \item Tested different hypothesis about human expansion out of Africa using simulations and Bayesian statistics. 
% \item \red{Used Bayesian Statistics to link genealogies to poSomething about Bayesian statistics... }
\end{rSubsection}

%------------------------------------------------
\begin{rSubsection}{ Convex Optimization}{Lausanne, Switzerland}{Advisor: Nisheeth Vishnoi (Theoretical Computer Science)}{June 2018 -- September 2018}
\item Participated in Summer@EPFL CS program at the \'Ecole polytechnique f\'ed\'erale de Lausanne (EPFL).
\item Designed and executed a project at interface of optimization, cosmology, and Riemannian geometry.
\item Developed manifold optimization algorithms to measure galaxy shapes from surface brightness profiles.
\item Used non-convex optimization techniques to mathematically show the high efficiency of my algorithm.
\end{rSubsection}
%------------------------------------------------
\begin{rSubsection}{General Relativity and Field Theory Honors Thesis}{Stanford, CA}{Advisor: Eva Silverstein (Cosmology)}{June 2017 -- June 2018}
\item Developed a framework for understanding scattering processes in manifolds by combining insights from quantum scattering theory, differential geometry, and partial differential equations. 
\item Applied framework to successfully resolve paradox of light waves traveling in Anti-de Sitter space. 
\item Simulated complex wave scattering processes using Mathematica. 
\item Presented work as my undergraduate Honors Thesis to the Stanford Physics Undergraduate Committee and at the Stanford Symposium of Undergraduate Research (SURPS).                        
\end{rSubsection}
\end{rSection}


%----------------------------------------------------------------------------------------
%   TEACHING EXPERIENCE SECTION
%----------------------------------------------------------------------------------------
\begin{rSection}{TEACHING EXPERIENCE} 
%
\begin{rSubsection}{Statistics Teaching Assistant at the University of Michigan}{Ann Arbor, MI}{}{}
\item \textbf{Courses:}
%
    \begin{itemize}[itemsep=-0.4em]
        \vspace*{-0.5em}
        \item Statistics 507: Data Science and Analytics using Python \hfill \textit{August 2020 -- Present}
    \end{itemize}
%
\end{rSubsection} 
%
\begin{rSubsection}{Physics Teaching Assistant at the University of Michigan}{Ann Arbor, MI}{}{}
\item \textbf{Courses:}
%
    \begin{itemize}[itemsep=-0.4em]
        \vspace*{-0.5em}
        \item Physics 136: Physics for the Life Sciences Laboratory I \hfill \textit{September 2019 -- December 2019}
        \item Physics 141: Elementary Laboratory I \hfill \textit{January 2020 -- April 2020}
    \end{itemize}
%
\item Guided students through a series of physics experiments including analysis of their measurements.
\end{rSubsection} 
%
\begin{rSubsection}{Physics Teaching Assistant at Stanford University}{Stanford, CA}{}{}
\item \textbf{Courses:}
%
    \begin{itemize}[itemsep=-0.5em]
        \vspace*{-0.5em}
        \item Physics 21: Mechanics, Fluids, and Heat \hfill \textit{September 2018 -- December 2018}
        \item Physics 70: Foundations of Modern Physics \hfill \textit{September 2017 -- December 2017}
    \end{itemize}
%
\item Designed and graded weekly problem sets, quizzes, and exams. 
\item Lead weekly problem-solving sessions aimed at reinforcing student's understanding of lecture. 
%
\end{rSubsection} 
%
\textbf{EPASA}: Tutored middle school student in math and English. \hfill {\em September 2016 -- June 2018}

\vspace*{-.15cm}
\textbf{Habla}: Tutored Stanford custodial staff in English 3 hours/week. \hfill {\em September 2014 -- June 2018}
\end{rSection}


%----------------------------------------------------------------------------------------
%   HONORS AND AWARDS SECTION
%----------------------------------------------------------------------------------------
\begin{rSection}{Honors and Awards} \itemsep -3pt \vspace*{-.25cm}
\item \textbf{Stanford Undergraduate Advising and Research Major Grant}: Grant recipient for research \hfill {\em 2017}

\item \textbf{Large Synoptic Survey Telescope Corporation Enabling Science Award}: Grant recipient for research \hfill {\em 2016}

\item \textbf{45th International Physics Olympiad}:  Bronze Medalist \hfill {\em 2014}
\end{rSection}
%----------------------------------------------------------------------------------------
% PROGRAMMING SKILLS SECTION
%----------------------------------------------------------------------------------------
\begin{rSection}{PROGRAMMING SKILLS}
%
Python, C/C++, \LaTeX, Mathematica, Unix shell, Git, R
\end{rSection}


%----------------------------------------------------------------------------------------
% PUBLICATIONS
%----------------------------------------------------------------------------------------
\begin{rSection}{PUBLICATIONS}
%
\vspace*{-1em}
\item Sanchez, J., Mendoza, I., Kirkby, D. P., \& Burchat, P. R. (2021). \textit{Effects of overlapping sources on cosmic shear estimation: Statistical sensitivity and pixel-noise bias.} arXiv preprint arXiv:2103.02078.
\end{rSection}


%----------------------------------------------------------------------------------------
% PUBLICATIONS
%----------------------------------------------------------------------------------------
\begin{rSection}{PRESENTATIONS}
%
\vspace*{-1em}
\item  \textit{Blending ToolKit Tutorial}, Ismael Mendoza, Dark Energy Science Collaboration (DESC) Summer 2020 Virtual Meeting, Chicago, IL. July 2020 (virtual)
%
\item \textit{The Blending Problem in Cosmology}, Ismael Mendoza, Physics Graduate Student Symposium 2020, Ann Arbor, MI. July 2020 (virtual)
%
\item \textit{BlendingToolKit: Walkthrough and Future Plans}, Ismael Mendoza, DESC Blending Working Group. July 2020 (virtual)
\end{rSection}



%%%%%%%%%%%%%%%%%%%%%%%%%%%%%%%%%%%%%%%%%%%%%%%%%%%%%%%%%%%%%%%%%%%%%%%%%%%%%%%%%%%%%%%%%%%%%%%%%%%%%%%%%%%%%%%%%%%%%%
\end{document}

% \begin{rSubsection}{Teaching Assistant for }{Stanford, CA}{Professor: Roger Romani}{September 2018 -– December 2018}
% \item Worked 20 hours/week to help teach this Physics course to Stanford undergraduates.   
% \item Helped students with topics such as quantum mechanics and special relativity during lecture and office hours. 
% \item 
% \end{rSubsection} 